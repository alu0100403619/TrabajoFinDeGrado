%
% ---------------------------------------------------
%
% Proyecto de Final de Carrera:
% Author: José Lucas Grillo Lorenzo <jlucas.gl@gmail.com>
% Capítulo: Conclusiones y trabajos futuros
% Fichero: Cap7_conclusiones.tex
%
% ----------------------------------------------------
%


\chapter{Conclusiones y trabajos futuros} \label{chap:conclusiones}  

\section{Conclusiones} \label{sec:conclusiones}

En el presente \acf{PFC}, se han abordado los asuntos relativos a las 
optimizaciones de bucles y su implementación, mediante las cláusulas \gang{} y \worker{}, 
en el contexto del estándar de programación heterogénea basado en directivas \OpenACC{}. 
No menos importante ha sido todo el trabajo de depuración y optmización del código que
compone el compilador, para obtener una versión de \accULL{} capaz de compilar 
satisfactoriamente los códigos evaluados. Así mismo es destacable también el trabajo
realizado por el autor para lograr elaborar un producto listo para usar a partir del
software del que está compuesto \accULL{}. 
Importante ha sido en esta labor el trabajo con las
herramientas de versionado y gestión de repositorio, sin las cuales las tareas de 
depuración y publicación de \textit{release} hubieran sido mucho más tediosas.
%XXX - Pasa revista a lo que has hecho:
En esta memoria de PFC se han revisado los logros conseguidos en la ejecución del 
proyecto. Relacionándolos directamente con los objetivos planteados inicialmente, podemos 
señalar como más significativos los siguientes logros: 
%(pon los más directos, los menos abstractos)

Se ha conseguido que \accULL{} compile y genere código correcto, 
comprobándolo en tiempo de ejecución para los niveles 0 y 1 del benchmark del EPCC 
\cite{URL::ACCepccB}, así como para los códigos \textit{27stencil} y \textit{le\_core} del 
nivel 2. Previo a la realización de este \ac{PFC} solo era posible compilar el nivel 0 y 
algunos códigos sencillos del 
nivel 1. El test \textit{le\_core} tampoco compilaba debido a diversos problemas con la 
\ac{TS}. Concretamente, no había soporte para la definición encadenada de tipos mediante 
\texttt{typedef}, y las opciones disponibles entonces estaban muy limitadas y daban serios 
problemas al volver a parsear el código, una vez volcado.

El grado de madurez alcanzado por \accULL{} es tal que a través de los experimentos 
realizados (véase Capítulo \ref{chap:resultados}) \accULL{} se ha convertido en una 
herramienta suficientemente madura para ofrecer resultados comparables en muchos casos a los 
que se pueden conseguir con herramientas comerciales.
No se pretende con ello decir que \accULL{} sea una herramienta acabada. En muchos casos 
somos conscientes de sus limitaciones; pueden aparecer bugs, el tiempo de compilación es 
en muchos casos elevado, el manejo de errores es mejorable, etc.
Entre estas limitaciones destaca el compilador (\yacf{}). Un compilador escrito en Python, 
un lenguaje interpretado, que tiene algunas carencias. Entre ellas una búsqueda en la 
Tabla de Símbolos ineficiente, o el hecho repetir varios ciclos de parseo y volcado de 
código en sucesivas etapas, hacen que \yacf{} sea uno de los puntos débiles de \accULL{}. 
En este sentido los ciclos de parseo-volcado nos han permitido desarrollar optimizaciones 
y transformaciones de código con relativa facilidad. No obstante es 
evidente la necesidad de estudiar sustituir el actual compilador \yacf{} basado en 
\textit{pycparser} \cite{Bendersky:pyc:2009}
por algún otro basado en herramientas con mayor soporte.

\section{Trabajos futuros} \label{sect:trabajo}

%Clasifícalos como 'cosas que vamos a acometer en breve' y 'cosas de mucha enjundia'.
Se clasifican a continuación, en orden de más prioritario a menos, las líneas abiertas de 
trabajos futuros que han de ser acometidos próximamente en el contexto de la 
investigación con el framework de compilación \accULL{}.

\begin{itemize}
\item Solucionar los casos de resultados incorrectos cuando haya que distribuir menos 
iteraciones que los \thread{}s de los que se disponga, al aplicar la planificación 
mediante las cláusulas \texttt{num\_gangs} y \texttt{num\_workers}, o en general al 
invocar a las cláusulas \gang{}/\worker{}.
\item Dar soporte para las reducciones de variables privadas en kernels de 2 o más 
dimensiones.
\item Evaluar la plataforma con más \benchmark{}s siguiendo el trabajo llevado a cabo para 
fortalecer su robustez. Como trabajo inmediato, está previsto evaluar los \textit{NAS 
Parallel \benchmark{}} \cite{Bailey:1991:NPB} verificando la correcta ejecución de los 
experimentos, y depurando a su vez los hipotéticos \textit{bugs} que surjan.
\item Implementar el soporte para las cláusulas \gang{}/\worker{} en regiones de tipo 
\texttt{kernels}, ya que actualmente solo están disponibles para regiones 
\texttt{parallel}.
\item Experimentar de forma intensiva con aplicaciones reales. Aprovechando las 
colaboraciones con el grupo del Dr. D. C. García Sanchez de la Universidad Complutense de 
Madrid, así como con 
el \ac{IAC} - a través del profesor Dr. D. F. Garzón, y el Dr. D. Antonio Dorta - este 
último miembro del grupo de 
investigación \GCAP{}.
\item Dar soporte para kernels 3D. 
\item Reemplazar \yacf{} por alguna alternativa más robusta como podría ser Cetus 
\cite{Lee:2003:CEC}, 
Rose \cite{Liao:2010:ESS}, Mercurium \cite{URL::Mercurium}, o LLVM \cite{Guntli:2011:AC}, 
como se ha discutido en la Sección \ref{sec:conclusiones}.
\end{itemize}
