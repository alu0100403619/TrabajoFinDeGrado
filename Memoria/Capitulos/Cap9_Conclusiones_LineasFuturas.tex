%
% ---------------------------------------------------
%
% Trabajo de Final de Grado:
% Author: Gonzalo Jesús García Martín <dracoyue@gmail.com>
% Capítulo: Fases
% Fichero: Cap8_Conclusiones_LineasFuturas.tex
%
% ----------------------------------------------------
%

\cleardoublepage
\chapter{Conclusiones y Lineas futuras}
\label{chap:conclusions}

	\section{Lineas futuras}
	
		Se contemplarán casos de uso que no se han implementado a la hora de crear la aplicación. Algunos perfiles ordenados según su necesidad son:
		
		\begin{itemize}
			\item Padres con hijos en distintos centros: En algún momento, alumnos con el mismo padre, madre o tutor legal asistirán a centros escolares distintos.
			\item Usuarios con distintos perfiles en el mismo colegio, por ejemplo, un profesor que de clases en el colegio al que asiste su hijo.
			\item Profesores que trabajan en más de un centro escolar: En este caso se contemplará profesores que impartan clases en más de un colegio.
		\end{itemize}
		
	\section{Conclusiones}
		La realización de este proyecto permite aplicar los conocimientos técnicos obtenidos durante los años de estudio, permitiendo la asimilación real de las competencias necesarias y la adquisición nuevos conocimientos.
		Esta experiencia da a conocer el grado de implicación que conlleva crear una aplicación totalmente funcional en un ámbito real. El análisis de otras aplicaciones similares permite concretar el estado del mercado con respecto al tipo de aplicación creada.
		
		\bigskip
		Programar una aplicación puede parecer sencillo, pero conlleva dedicación, tiempo, trabajo y esfuerzo.