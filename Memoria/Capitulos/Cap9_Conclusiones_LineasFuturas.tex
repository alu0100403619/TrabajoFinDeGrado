%
% ---------------------------------------------------
%
% Trabajo de Final de Grado:
% Author: Gonzalo Jesús García Martín <dracoyue@gmail.com>
% Capítulo: Fases
% Fichero: Cap8_Conclusiones_LineasFuturas.tex
%
% ----------------------------------------------------
%

\cleardoublepage
\chapter{Conclusiones y Trabajos futuros}
\label{chap:conclusions}

	En este capítulo se revisarán los posibles trabajos por hacer en \CollegeApp, ya que toda aplicación debe estar dispuesta a ser mejorada. También se hablará de las conclusiones y experiencias adquiridas a lo largo de la realización de este proyecto que ha sido especialmente enriquecedor.

	\section{Trabajos futuros}
	
		Se contemplarán casos de uso que no se han implementado a la hora de crear la aplicación. Algunos perfiles ordenados según su dificultad de desarrollo son:
		
		\begin{itemize}
			\item Padres con hijos en distintos centros: en algún momento, alumnos con el mismo padre, madre o tutor legal podrían asistir a centros escolares distintos.
			\item Usuarios con distintos perfiles en el mismo colegio, por ejemplo, un profesor que imparta clases en el colegio al que asiste su hijo.
			\item Profesores que trabajan en más de un centro escolar: En este caso se contemplará profesores que impartan clases en más de un colegio.
		\end{itemize}
		
		También se contemplarán otras funcionalidades, estas están ordenadas en función de su dificultad de implementación:
		\begin{itemize}
			\item Búsqueda de contactos: El usuario podrá buscar un contacto concreto introduciendo su nombre en un cuadro de búsqueda.
			\item Envío de boletines de notas: Proporcionará a los profesores la funcionalidad de enviar las notas de sus alumnos a los padres de éstos.
			\item Envío de archivos a través de la aplicación: Permitirá a los usuarios compartir archivos.
		\end{itemize}
		
	\section{Conclusiones}
		La realización de este proyecto nos ha permitido aplicar los conocimientos técnicos obtenidos durante los años de estudio, permitiendo la asimilación real de las competencias necesarias y la adquisición nuevos conocimientos.
		Esta experiencia da a conocer el grado de implicación que conlleva crear una aplicación totalmente funcional en un ámbito real. El análisis de otras aplicaciones similares permite concretar el estado del mercado con respecto al tipo de aplicación creada.
		
		\bigskip
		Programar una aplicación puede parecer sencillo, pero conlleva dedicación, tiempo, trabajo y esfuerzo.