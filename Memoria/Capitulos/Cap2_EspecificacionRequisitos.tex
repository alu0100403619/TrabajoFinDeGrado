%
% ---------------------------------------------------
%
% Trabajo de Final de Grado:
% Author: Gonzalo Jesús García Martín <dracoyue@gmail.com>
% Capítulo: Especificación de Requisitos
% Fichero: Cap2_EspecificacionRequisitos.tex
%
% ----------------------------------------------------
%

\cleardoublepage
\chapter{Especificación de Requisitos}
\label{chap:requirements}

\section{Funcionalidades}

\section{Pantallas}
La aplicación tendrá una pantalla de bienvenida con tres pestañas:
	\begin{itemize}
		\item Bienvenido: Aquí se mostrará los botones para acceder a la aplicación y para el registro de usuarios.
		\item Ayuda: En esta pestaña se obtendrá una pequeña ayuda sobre la aplicación.
		\item Sobre el Autor: Se mostrará la información sobre el autor de la aplicación.
	\end{itemize}
	En la pantalla de registro se le pedirá al usuario una serie de datos a cumplimentar según su rol: 
	\begin{itemize}
		\item Alumno:
			\begin{itemize}
				\item Nombre.
				\item Apellidos.
				\item Teléfono.
				\item Centro Escolar.
				\item Grupo y Curso al que pertenece: 1A, 2C, 3B... Solo podrá poner un curso y un grupo.
				\item Correo electrónico.
				\item Contraseña.
			\end{itemize}
		\item Profesor:
			\begin{itemize}
				\item Nombre.
				\item Apellidos.
				\item Teléfono.
				\item Centro Escolar.
				\item Grupo y Curso al que pertenece: 1A, 2C, 3B... Podrá poner varios cursos y grupos separados por comas.
				\item Correo electrónico.
				\item Contraseña.
			\end{itemize}
		\item Padre/Madre/Tutor Legal:
			\begin{itemize}
				\item Nombre.
				\item Apellidos.
				\item Teléfono.
				\item Centro Escolar al que asiste el alumno del que es padre.
				\item Correo electrónico.
				\item Contraseña.
				\item Nombre del alumno del que es padre.
				\item Apellidos del alumno del que es padre.
				\item Correo electrónico del alumno del que es padre.
				\item Centro escolar del alumno del que es padre.
				\item Curso y grupo del alumno del que es padre 1A, 2C, 3B... Solo podrá poner un curso y un grupo.
			\end{itemize}
	\end{itemize}
	En la de acceso se le pedirá su dirección de correo y la contraseña, el usuario tendrá, además, la posibilidad de solicitar que se le recuerde la contraseña por correo.
	\bigskip
	Una vez se accede a la aplicación a través del e-mail y contraseña, ésta se encarga de clasificar al usuario según el rol con el que se halla registrado, rellenando los datos de forma correcta según el perfil seleccionado. A continuación se describirá cada una de las pestañas en sus respectivos roles:
	\bigskip
	\begin{itemize}
		\item Alumnos:
			\begin{itemize}
				\item Alumnos: Esta pestaña contendrá los datos de los compañeros de clase del usuario.
				\item Profesores: Ésta se rellenará con los datos de los profesores que le dan clase al alumno.
				\item Notificaciones: En ésta se podrán encontrar los nombres de quienes envíen mensajes al usuario.
			\end{itemize}
		\item Profesores:
			\begin{itemize}
				\item Profesores: Aquí podremos encontrar los compañeros que dan clase a los grupos a los que el usuario da clase.
				\item Padres: Esta pestaña se rellenará con los padres de los alumnos a los que el profesor da clase.
				\item Alumnos: Aquí podremos encontrar a los alumnos a los que el profesor da clase
				\item Notificaciones: En ésta se podrán encontrar los nombres de quienes envíen mensajes al usuario.
			\end{itemize}
		\item Padres:
			\begin{itemize}
				\item Padres: En esta pestañ se encontrarán los padres de los compañeros de clase del hijo/s del usuario.
				\item Profesores: Aquí estarán los profesores que dan clase al hijo/s del usuario.
				\item Notificaciones: En ésta se podrán encontrar los nombres de quienes envíen mensajes al usuario.
			\end{itemize}
	\end{itemize}
									
	También tendremos una opción para enviar circulares a otros usuarios de la aplicación pudiendo adjuntar archivos. Los padres podrán concertar citas con los profesores y viceversa.
	Al seleccionar cualquiera de los usuarios que aparecen en las listas, se lanzará un ``chat'' para que los usuarios se comuniquen entre ellos. Al enviar un mensaje éste se almacenará en los servicios en la nube, en la tabla destinada a los mensajes. Esta tabla contendrá el mensaje, el destinatario y el remitente. Cuando el dispositivo reciba el mensaje, lo borrará de la tabla para no sobrecargarla.
	
	\bigskip
	Al hacer una selección larga, se podrán ver los datos del usuario que se selecciona.

	Todos los datos estarán almacenados en la nube, en concreto en los servicios de Firebase\cite{6:firebase:online} que es propiedad de Google\cite{18:google:online}.

\section{Sistemas de la Aplicación}