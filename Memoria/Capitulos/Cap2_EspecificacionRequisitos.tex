%
% ---------------------------------------------------
%
% Trabajo de Final de Grado:
% Author: Gonzalo Jesús García Martín <dracoyue@gmail.com>
% Capítulo: Especificación de Requisitos
% Fichero: Cap2_EspecificacionRequisitos.tex
%
% ----------------------------------------------------
%

\cleardoublepage
\chapter{Especificación de Requisitos}
\label{chap:requirements}

	\section{Funcionalidades}
		A continuación se describen las funcionalidades de la aplicación:
		\begin{itemize} 
			\item {\bf Registro}: Tras rellenar el formulario y accionar el botón de registro, \CollegeApp enviará los datos a la Firebase. Éstos serán almacenados en la base de datos, en sus tablas correspondientes. También creará al usuario correspondiente para que tenga acceso a la aplicación.
			\item {\bf Acceso}: El usuario introducirá su dirección de correo electrónico y contraseña en los campos destinados a ello. El dispositivo móvil se autenticará contra el servicio de Firebase y si todo es correcto permitirá el acceso.
			\item {\bf Recuerdo de Contraseña}: La aplicación solicitará al servicio en la nube una contraseña temporal para que el usuario pueda acceder, tras lo cual tendrá que editar su contraseña en \CollegeApp.
			\item {\bf Contactos}: Se recuperarán todos los usuarios relacionados con la persona que esté usando la aplicación. Esto permitirá que se puedan comunicar.
			\item {\bf chat}: Los usuarios se podrán enviar mensajes entre sí, éstos se almacenaran en Firebase para ser recuperados por la aplicación y luego borrados de la nube. La comunicación será del tipo alumno-profesor, padre-profesor y viceversa.
			\item {\bf Notificaciones}: \CollegeApp recuperará de la base de datos todos los mensajes que tenga el usuario, para después borrarlos de su almacenamiento en la nube.
			\item {\bf Circulares}: Aviso que se mandará a todos los integrantes de una clase, e incluso de varias. Estos mensajes se almacenarán en la nube y serán recuperados por la aplicación, que los eliminará de su almacenamiento en internet.
			\item {\bf Citas}: Una vez concertada una cita con un profesor, se podrá mandar al calendario de Google para tenerla registrada.
			\item {\bf Visualización de Datos}: El usuario podrá visualizar la información de sus contactos con una selección larga en uno de ellos.
			\item {\bf Edición de datos}: Se podrá editar los datos en el dispositivo móvil, tras lo cual se actualizarán en la base de datos almacenada en la nube. 
		\end{itemize}

	%\section{Pantallas}
	\section{Actividades}
		La aplicación tendrá una pantalla de bienvenida con tres pestañas:
		\begin{itemize}
			\item Bienvenido: Aquí se mostrará los botones para acceder a la aplicación y para el registro de usuarios.
			\item Ayuda: En esta pestaña se obtendrá una pequeña ayuda sobre la aplicación.
			\item Sobre el Autor: Se mostrará la información sobre el autor de la aplicación.
		\end{itemize}
		
		En la pantalla de registro se le pedirá al usuario una serie de datos a cumplimentar según su rol: 
		\begin{itemize}
			\item Alumno:
				\begin{itemize}
					\item Nombre.
					\item Apellidos.
					\item Teléfono.
					\item Centro Escolar.
					\item Grupo y Curso al que pertenece: 1A, 2C, 3B... Solo podrá poner un curso y un grupo.
					\item Correo electrónico.
					\item Contraseña.
				\end{itemize}
			\item Profesor:
				\begin{itemize}
					\item Nombre.
					\item Apellidos.
					\item Teléfono.
					\item Centro Escolar.
					\item Grupo y Curso al que pertenece: 1A, 2C, 3B... Podrá poner varios cursos y grupos separados por comas.
					\item Correo electrónico.
					\item Contraseña.
				\end{itemize}
			\item Padre/Madre/Tutor Legal:
				\begin{itemize}
					\item Nombre.
					\item Apellidos.
					\item Teléfono.
					\item Centro Escolar al que asiste el alumno del que es padre.
					\item Correo electrónico.
					\item Contraseña.
					\item Nombre del alumno del que es padre.
					\item Apellidos del alumno del que es padre.
					\item Correo electrónico del alumno del que es padre.
					\item Centro escolar del alumno del que es padre.
					\item Curso y grupo del alumno del que es padre 1A, 2C, 3B... Solo podrá poner un curso y un grupo.
				\end{itemize}
		\end{itemize}
		
		En la pantalla de acceso se le pedirá su dirección de correo y la contraseña, el usuario tendrá, además, la posibilidad de solicitar que se le recuerde la contraseña por correo.
		
		\bigskip
		Una vez se accede a la aplicación a través del correo electrónico y contraseña, ésta se encarga de clasificar al usuario según el rol con el que se haya registrado, rellenando los datos de forma correcta según el perfil seleccionado. A continuación se describirá cada una de las pestañas en sus respectivos roles:
		\bigskip
		\begin{itemize}
			\item Alumnos:
				\begin{itemize}
					\item Alumnos: Esta pestaña contendrá los datos de los compañeros de clase del usuario.
					\item Profesores: Ésta se rellenará con los datos de los profesores que le dan clase al alumno.
					\item Notificaciones: En ésta se podrán encontrar los nombres de quienes envíen mensajes al usuario.
				\end{itemize}
			\item Profesores:
				\begin{itemize}
					\item Profesores: Aquí podremos encontrar los compañeros que dan clase a los grupos a los que el usuario da clase.
					\item Padres: Esta pestaña se rellenará con los padres de los alumnos a los que el profesor da clase.
					\item Alumnos: Aquí podremos encontrar a los alumnos a los que el profesor da clase
					\item Notificaciones: En ésta se podrán encontrar los nombres de quienes envíen mensajes al usuario.
				\end{itemize}
			\item Padres:
				\begin{itemize}
					\item Padres: En esta pestaña se encontrarán los padres de los compañeros de clase del hijo/s del usuario.
					\item Profesores: Aquí estarán los profesores que dan clase al hijo/s del usuario.
					\item Notificaciones: En ésta se podrán encontrar los nombres de quienes envíen mensajes al usuario.
				\end{itemize}
		\end{itemize}
										
		Los profesores tendrán una opción para enviar circulares a otros usuarios de la aplicación. Los padres podrán concertar citas con los profesores y viceversa.
		Al seleccionar cualquiera de los usuarios que aparecen en las listas, se lanzará un ``chat'' para que los usuarios se comuniquen entre ellos. Al enviar un mensaje éste se almacenará en los servicios en la nube, en la tabla destinada a los mensajes. Esta tabla contendrá el mensaje, el destinatario y el remitente. Cuando el dispositivo reciba el mensaje, lo borrará de la tabla para no sobrecargarla.
		
		\bigskip
		Al hacer una selección larga, se podrán ver los datos del usuario que se selecciona.
	
		Todos los datos estarán almacenados en la nube, en concreto en los servicios de Firebase\cite{6:firebase:online} que es propiedad de Google\cite{18:google:online}.

	\section{Sistemas de la Aplicación}
		\begin{itemize}
			\item Sistema de Registro: Al usuario se le podrá pedir los siguientes datos, según el rol con el que se registre:
			\begin{itemize}
				\item Nombre y Apellidos.
				\item Rol con el que se va a registrar, ya que en función de un rol u otro se le pedirá distintos datos.
				\item Número de teléfono.
				\item Nombre del centro con el que va a hacer el registro.
				\item Clase y grupo.
				\item E-mail y contraseña.
				\item Nombre y Apellidos del alumno.
				\item Clase en la que está el alumno.
				\item Centro en el que está el alumno.
				\item Curso en el que está el alumno.
			\end{itemize}
			\item Sistema de ``acceso'': Se le pedirá al usuario que ingrese su identificación, es decir, su e-mail y contraseña.
			\item Sistema de comunicación entre alumnos, profesores y padres/madres/tutores legales: Tanto los profesores como los padres/madres/tutores legales se podrán comunicar entre sí con un servicio tipo ``chat'', para informar de alguna incidencia, mensaje o notificación. Este servicio enviará los mensajes a través de internet usando los servicios de ``firebase''.
			\item Sistema de citas: Se podrá concertar citas con los padres/madres/Tutores Legales, así como éstos podrán solicitar reuniones con los profesores, estas citas se podrán añadir al calendario de Google.
			\item Lista de contactos, organizada por grupos desplegables, por ejemplo los profesores tendrán la lista organizadas por las clases que impartan en el centro.
		\end{itemize}
	
		Esto permitirá establecer un sistema de comunicaciones entre los profesores y los padres de alumnos.

	\section{Roles}
		\begin{itemize}
			\item Alumnos: Se podrá comunicar con los profesores, pudiendo solicitar citas y recibir circulares.
			\item Profesores: Se podrá comunicar con los alumnos y padres, pudiendo solicitar citas y enviar circulares. Los padres y alumnos en la lista de contactos estarán organizados por las clases que impartan en el centro.
			\item Padre/Madre/Tutor Legal: Se podrá comunicar con los profesores, pudiendo recibir circulares y solicitar citas. Los profesores en la lista de contactos estarán organizados las clases que impartan a los alumnos con los que el progenitor esté relacionado.
		\end{itemize}
