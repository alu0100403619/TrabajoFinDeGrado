%
% ---------------------------------------------------
%
% Trabajo de Final de Grado:
% Author: Gonzalo Jesús García Martín <dracoyue@gmail.com>
% Capítulo: Fases
% Fichero: Cap6_Fases.tex
%
% ----------------------------------------------------
%

\cleardoublepage
\chapter{Fases}
\label{chap:phases}

	Las fases de diseño del proyecto fueron las siguientes:
	
%	\begin{enumerate}
%		\item Identificar las funcionalidades de la aplicación, es decir, la especificación de requisitos.
%		\item Escoger una funcionalidad concreta.
%		\item Implementar esa parte de la aplicación.
%		\item Probar el código resultante y corregir los errores que pueda ocasionar.
%		\item Corrección de errores y adaptación de otras partes de la aplicación para la correcta integración de la nueva funcionalidad.
%		\item Si se encuentra una nueva funcionalidad añadirla a las ya identificadas.		
%		\item se seguirán estas indicaciones hasta haber implementado todos los requisitos de la aplicación.
%	\end{enumerate}

	\bigskip
	Primero se identifican las funcionalidades de la aplicación, es decir, la especificación de requisitos. Se escoge un requisito concreto de los que han sido identificados, se implementa la parte de la aplicación que corresponde con la funcionalidad escogida. Se prueba el código resultante y se corrigen los errores que puedan surgir en esta implementación, también se adaptan partes del código para que trabajen en conjunto con la nueva funcionalidad y se vuelve a la corrección de errores que puedan surgir. Por último se escoge otro requisito que implementar. Si se encuentra un nuevo requisito o funcionalidad que deba ser programado, se añadirá a las ya existentes.
	
	\bigskip
	Se siguen estas indicaciones hasta que la aplicación esté completa, es decir, hasta que no queden requisitos que sin cumplir.