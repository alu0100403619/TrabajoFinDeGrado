%
% ---------------------------------------------------
%
% Trabajo de Final de Grado:
% Author: Gonzalo Jesús García Martín <dracoyue@gmail.com>
% Capítulo: Objetivos
% Fichero: Cap3_Objetivos.tex
%
% ----------------------------------------------------
%

\cleardoublepage
\chapter{Objetivos}
\label{chap:objetives}

	En este capítulo se pasará a describir tanto los objetivos generales como los específicos.
	
	%%%%%%%%%%%%%%%%%%%%%%%%%%%%%%%%%%%%%%%%%%%%%%%%%%%%%%%%%%%%%%%%%%%%%%%%%%%%%%%%%%%%%%%%%%%%%%%%%%%%%
	%Cuando elaboramos un proyecto, hay que definir sus objetivos, que pueden ser de tres tipos: 
	%
	%Objetivos generales
	%
	%Los objetivos generales corresponden a las finalidades genéricas de un proyecto o entidad.
	%
	%No señalan resultados concretos ni directamente medibles por medio de indicadores pero si que expresan el propósito central del proyecto.
	%Tienen que ser coherentes con la misión de la entidad. 
	%
	%Los objetivos generales se concretan en objetivos específicos.
	%
	%Objetivos específicos
	%
	%Se derivan de los objetivos generales y los concretan, señalando el camino que hay que seguir para conseguirlos. Indican los efectos
	%específicos que se quieren conseguir aunque no explicitan acciones directamente medibles mediante indicadores. 
	%
	%Objetivos operativos
	%
	%Concreten los objetivos específicos. Son cuantificables, medibles mediante indicadores y directamente verificables. Así nos permiten
	%hacer seguimiento y evaluación del grado de cumplimiento de los efectos que se quieren conseguir con los objetivos específicos. 
	%%%%%%%%%%%%%%%%%%%%%%%%%%%%%%%%%%%%%%%%%%%%%%%%%%%%%%%%%%%%%%%%%%%%%%%%%%%%%%%%%%%%%%%%%%%%%%%%%%%%%
	
	\section{Objetivos Generales}
		Se pretende crear una aplicación en Android como trabajo de fin de grado, utilizando las nuevas tecnologías como los teléfonos inteligentes y los servicios gratuitos de internet conocidos como ``Nube''.
	
	\section{Objetivos Específicos}
		Como objetivos específicos
		