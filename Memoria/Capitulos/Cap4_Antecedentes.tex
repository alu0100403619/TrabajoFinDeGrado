%
% ---------------------------------------------------
%
% Trabajo de Final de Grado:
% Author: Gonzalo Jesús García Martín <dracoyue@gmail.com>
% Capítulo: Objetivos
% Fichero: Cap4_Antecedentes.tex
%
% ----------------------------------------------------
%

\cleardoublepage
\chapter{Antecedentes}
\label{chap:record}

	Con los nuevos avances tecnológicos se ha vuelto popular el uso de las comúnmente denominada ``apps'' en los dispositivos móviles. Hay aplicaciones para todos los gustos, desde lo mas básico como aprender a cocinar o aplicaciones de comunicación, hasta de lo más importante como por ejemplo consultar la información de la cuenta del banco e incluso hacer operaciones con ella.
	
	\bigskip
	Se ha observado que en el ámbito docente hay una falta de comunicación entre padres y profesores. También el aumento de problemas como el acoso, fracaso escolar o incluso problemas de ámbito familiar influyen en los alumnos. Por eso los hasta ahora recursos tradicionales no eran suficientes. Notas escritas y reuniones no son mas que informes puntuales de un progreso continuo que puede decaer sin aviso previo.
	
	\bigskip
	Por eso se presenta una herramienta que intenta establecer un flujo de información continua sin comprometer los datos personales de los usuarios tales como el teléfono o el correo. Así éstos se sentirán cómodos a la hora de comunicarse de forma segura. Se entiende que es un esfuerzo extra para los equipos docentes, pero permitirá que haya una mejor comunicación para resolver problemas inesperados y actuar de forma casi inmediata.