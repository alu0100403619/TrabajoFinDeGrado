%
% ---------------------------------------------------
%
% Trabajo de Final de Grado:
% Author: Gonzalo Jesús García Martín <dracoyue@gmail.com>
% Capítulo: Introducción
% Fichero: Cap7_DesarrolloApp.tex
%
% ----------------------------------------------------
%

\cleardoublepage
\chapter{Desarrollo de la Aplicación}
\label{chap:developing}

	En este capítulo se expondrá la implementación de las clases más relevantes para la aplicación.
	Sería demasiado extenso explicar en profundidad cada uno de los elementos que componen la aplicación.
	La figura \ref{fig:ClassDiagram} presenta un diagrama de clases de \CollegeApp\ mientras que en el apéndice \ref{chap:activitiesdiagram} muestra el diagrama de actividades(figura \ref{fig:ActivitiesDiagram}).
	
	\bigskip
	Todo el código de \CollegeApp\ \cite{28:mirepo:online} está disponible públicamente en el repositorio de github.
	
	\section{Clase {\ttfamily Student}} \label{sec:Student}
	
		Esta es la clase encargada de almacenar los datos de los alumnos que se obtienen desde el {\it proveedor de servicios}.
		El constructor obtiene los datos desde un {\it HashMap} \cite{10:hashmap:online} que es el objeto que devuelven las consultas a la base de datos. También se ha implementado la interfaz {\it Parcelable} \cite{11:parcelable:online} que es la que permite compartir objetos de una clase entre {\it Activities}. 
		En el listado \ref{code:StudentJava} se puede observar la implementación de la clase.
		
		\lstinputlisting[float=h!, language=Java,caption={Ejemplo de la clase Student.},label={code:StudentJava}]{Code/Student.java}
		
		\bigskip
		Entre las líneas 19 y 25 se observa el método {\ttfamily writeToParcel()}, que implementa la forma de introducir datos de un objeto de la clase {\it Parcelable}.
		
		El método {\ttfamily readFromParcel()} (líneas de la 26 a la 30) muestra como se obtienen los datos de esta clase. En este método, los atributos deben estar en el mismo orden que el en {\ttfamily writeToParcel()}.
		
		De la línea 31 a la 36, se se muestra como crear un objeto de la clase {\it Parcelable}.
	
	\section{Clase {\ttfamily Father}}
	
		Esta es la clase encargada de almacenar los datos de los usuarios con el rol de padre que se obtienen desde la {\it nube}.
		Los atributos de la clase son: name, lastname, mail, telephone, dni, childrens ({\it ArrayList} \cite{12:arraytist:online} de la clase {\ttfamily Student} que son hijos del usuario) y rol.
		Funciona igual que la clase {\ttfamily Student} (sección \ref{sec:Student}).
				
		\bigskip
		Los métodos más importantes en esta clase son {\ttfamily getClassrooms()} que devuelve un ArrayList con las clases en las que están registrados los hijos y {\ttfamily getSchools()} que devuelve otro ArrayList con los colegios a los que asisten los hijos.
	
	\section{Clase {\ttfamily Teacher}}
	
		Esta es la clase encargada de almacenar los datos de los usuarios con el perfil de profesor que se obtienen desde la {\it nube}.
		Los atributos de la clase son name, lastname, mail, telephone, dni y rol.
		Funciona igual que la clase {\ttfamily Student} (sección \ref{sec:Student}).
		
		El constructor obtiene los datos desde un {\it HashMap} \cite{10:hashmap:online} que es el objeto que devuelven las consultas a la base de datos. También se ha implementado la interfaz {\it Parcelable} \cite{11:parcelable:online} que es la que permite compartir objetos de una clase entre {\it Activities}. 
		
	\section{Clase {\ttfamily Message}}
	
		Esta es la clase encargada de almacenar los datos de los mensajes que se obtienen desde la {\it nube}.
		A continuación se enumeran los atributos de la clase:
		
		\begin{itemize}
			\setlength{\itemsep}{1pt}
			\setlength{\parskip}{0pt}
			\setlength{\parsep}{0pt}
			\item {\it dniRemitter}: D.N.I del remitente del mensaje.
			\item {\it remitter}: Nombre del remitente del mensaje, se usa para mostrar el nombre en {\ttfamily NotificationsActivity} (sección \ref{sec:notifications}).
			\item {\it message}: Mensaje que envían los usuarios.
			\item {\it rolRemmiter}: Perfil del remitente, se usa para colorear las notificaciones.
			\item {\it date}: fecha del mensaje.
			\item {\it destinatario}: Este campo solo está presente en la base de datos. Se usa para recuperar los mensajes.
		\end{itemize}
		
		Funciona igual que la clase {\ttfamily Student} (sección \ref{sec:Student}).
		
	\section{Clase {\ttfamily MessageSQLHelper}}
		Cada uno de los mensaje enviados entre los usuarios está almacenado en el dispositivo, en una base de datos local {\it SQLite}\cite{13:sqlite:online}.
		
		\bigskip
		La tabla {\ttfamily messages} almacena los mensajes que se envían desde el dispositivo. Sus atributos son:
		
		\begin{itemize}
			\setlength{\itemsep}{1pt}
			\setlength{\parskip}{0pt}
			\setlength{\parsep}{0pt}
			\item {\it idConversation}: Identificación de la conversación.
			\item {\it dniRemitter}: D.N.I del remitente.
			\item {\it remitter}: Nombre del remitente.
			\item {\it message}: Mensaje. 
			\item {\it day, month, year, hour, minute}: Datos del envío del mensaje.
		\end{itemize}
		
		\bigskip
		La tabla {\ttfamily conversations} almacena el identificador de la conversación entre dos usuarios.
		Sus atributos son:
		
		\begin{itemize}
			\setlength{\itemsep}{1pt}
			\setlength{\parskip}{0pt}
			\setlength{\parsep}{0pt}
			\item {\it id}: Identificación de la conversación.
			\item {\it dniSender}: D.N.I del usuario que envía el mensaje.
			\item {\it dniRemitter}: D.N.I del remitente.
		\end{itemize}
		
		En el listado \ref{code:MessageDBJava} se muestra la implementación de la clase.
		
		\lstinputlisting[float = h!,language=Java,caption={Ejemplo de la clase MessageSQLHelper.},label={code:MessageDBJava}]{Code/MessageSQLHelper.java}
		
		Desde la línea 6 hasta la 24, se muestra el método {\ttfamily onCreate()} donde se crean las las tablas descritas anteriormente. Para cada tabla se crea una variable de tipo cadena ({\it String}) con la definición cada uno de sus atributos. Después se ejecuta la función {\ttfamily execSQL()} y se le pasa como parámetro la variable creada.
		
	\section{\ttfamily TabActivityes}
		Cuando el usuario accede a \CollegeApp\ se le muestra una pantalla con pestañas donde se hallan sus contactos. Cada pestaña es una actividad distinta a causa del problema \ref{prob:8}. Al cambiar de pestaña se cambia de actividad y la consulta deja de ejecutarse.
		Esta solución ha sido la principal forma de construir la aplicación y solventar los problemas principales.
		
		\bigskip
		En el listado \ref{code:studentTab} se muestra como se crean las pestañas en las que se mostrarán las listas de contactos. En la línea 5 se obtiene el elemento donde se añadirán las pestañas, llamado {\ttfamily TabHost} \cite{25:tabhost:online}. Se crea el objeto que lanzará la actividad deseado (línea 10) y se añade al {\ttfamily TabHost}.
		
		\lstinputlisting[float = h!,language=Java,caption={Ejemplo de la clase StudentTabActivity.},label={code:studentTab}]{Code/StudentTabActivity.java}
		
		\bigskip
		La clase {\ttfamily StudentActivity} (listado \ref{code:studentActivity}) muestra un ejemplo de la interacción con la base de datos. En la línea 8 se puede ver la consulta que obtiene los alumnos que asisten a la misma clase que el usuario. A esta consulta se le añade un oyente que conecta con la base de datos y espera por si se modifican los datos. Se crea el objeto de la clase correspondiente en la línea 13 y se añade a un {\it ArrayList} con el que se rellenará la lista de contactos.
		
		\lstinputlisting[float = h!,language=Java,caption={Ejemplo de la clase StudentActivity.},label={code:studentActivity}]{Code/StudentActivity.java}
	
		\bigskip
		En listado \ref{code:StudentRegisterJava} se muestra como se envía información a la base de datos. Al registrar al usuario se observa como existe una función que comprueba que todos los datos del formulario sean correctos (línea 7, {\ttfamily haveEmptyFields()}). Para permitir el acceso a \CollegeApp\ hay que crear al usuario como se muestra en la línea 8. Si se crea al usuario de forma satisfactoria, se procede a enviar la información a {\it Firebase}. Se crea un {\it HashMap} y se introducen los datos del usuario con el método {\ttfamily put()} (líneas 11 a 14). Se selecciona un identificador único (línea 15) y se envía a la base de datos. {\ttfamily studentRef} es la referencia a la tabla en la que se introducirán los datos, {\ttfamily child(uuid)} es la función que selecciona la fila correspondiente al usuario y {\ttfamily setValue(aluMap)} permite almacenar los datos que contiene el {\it HashMap} (línea 16).
		
		\lstinputlisting[float = h!,language=Java,caption={Ejemplo de la clase StudentRegisterActivity.},label={code:StudentRegisterJava}]{Code/StudentRegisterActivity.java}
	
	\section{Problemas}
		Al diseñar una aplicación siempre surgen problemas inesperados en los que hay que agudizar el ingenio para resolverlos.
		
		En este capítulo se expondrán los problemas ocurridos a la hora de programar, se irán enumerando y explicando la forma de resolverlos.
		
		\begin{enumerate}
			\item {\bf Android Studio no render target}
			\begin{itemize}
				\item {\bf Solución}:
				\begin{enumerate}
					\item Asegúrese de que hay un dispositivo real o virtual seleccionado.
					\item Tener una versión de {\it Android} seleccionada.
					\item Versiones necesarias para el desarrollo dela aplicación en {\it Android} instaladas.
					\item Crear un nuevo {\it Dispositivo Virtual} ({\it AVD}).
					% Meter dibujo de AVD
					\item Reiniciar Android Studio.
				\end{enumerate}
			\end{itemize}
			
			\item {\bf Fallo al encontrar com.android.support:appcompat-v7:16.+}
			\begin{itemize}
				\item {\bf Solución}: Actualizar en las dependencias del archivo {\ttfamily build.gradle}, en el directorio {\ttfamily app} que está en la raíz del proyecto, la librería deberá ser la última versión disponible. Este cambio se realiza de forma manual.
			\end{itemize}
			
			\item {\bf Fallo al encontrar Java}
			\begin{itemize}
				\item {\bf Solución}: Si java ya está instalado, averigüe el directorio. Una vez hecho esto necesita volver a establecer la variable de ámbito indicando la localización correcta. Seleccione {\ttfamily Iniciar \textgreater Equipo \textgreater Propiedades \textgreater Configuración Avanzada del Sistema}.
				Entonces abra la pestaña {\ttfamily Opciones Avanzadas \textgreater Variables de Entorno} y añada una nueva variable de sistema llamada {\ttfamily JAVA\_HOME} que  tenga como valor la dirección del directorio donde tenga instalado el {\it JDK}, por ejemplo, {\ttfamily C:{\textbackslash}Program Files{\textbackslash}Java{\textbackslash}jdk1.7.0\_21}. 
			\end{itemize}
			\item {\bf Duplicidad en las dependencias de los paquetes}
			\begin{itemize}
				\item {\bf Solución}: Si se tiene un error de compilación sobre archivos duplicados, se puede excluir esos ficheros añadiendo la directiva {\ttfamily packagingOptions} al archivo {\ttfamily build.gradle} que está en el directorio {\ttfamily app}. Como se muestra en el listado \ref{code:duplicidadbuild}.
				
				\lstinputlisting[float = h!, language=Java,caption={Solución a la duplicidad en {\ttfamily build.gradle}.}, label={code:duplicidadbuild}]{Code/duplicidadbuild.gradle}
			\end{itemize}
			
			\item {\bf Fallo al encontrar android.support.v13}
			\begin{itemize}
				\item {\bf Solución}: Añadir en las dependencias del archivo {\ttfamily build.gradle}, en el directorio {\ttfamily app} que está en la raíz del proyecto, la orden {\ttfamily compile 'com.android.support:support-v13:21.+'} de forma manual. Como se puede analizar en el listado \ref{code:supportv13}.
				
				\noindent
				\lstinputlisting[float = h!, language=Java,caption={Solución al problema supportV13.},label={code:supportv13}]{Code/support13build.gradle}
			\end{itemize}
			
			\item {\bf Arregle Gradle (Fix Gradle)}
			\begin{itemize}
				\item {\bf Solución}:
				\begin{enumerate}
					\item En el archivo {\ttfamily buil.gradle} que está en el directorio raíz del proyecto, añadir la dependencia de forma manual: {\ttfamily classpath 'com.android.tools.build:gradle:1.0.0' }. Listado \ref{code:buildGradle}.
					\item En el archivo {\ttfamily buil.gradle}, que está en el directorio {\ttfamily app} en la raíz del proyecto, añadir dentro de {\ttfamily release} de forma manual, listado \ref{code:appBuildGradle}.
					
				\end{enumerate}
				\noindent
				\lstinputlisting[float = h!,language=Java,caption={Solución en {\ttfamily build.gradle}.},label={code:buildGradle}]{Code/buildgradle.gradle}
				\lstinputlisting[float = h,language=Java,caption={Solución en {\ttfamily app/build.gradle}.},label={code:appBuildGradle}]{Code/appbuildgradle.gradle}
			\end{itemize}
			
			\item {\bf SDK no encontrado}
			\begin{itemize}
				\item {\bf Solución}: Si AndroidStudio no encuentra el {\it SDK} y está instalado seleccionar {\ttfamily Windows \textgreater Preferencias \textgreater Android \textgreater Localización SDK} y establecer el directorio donde se tiene instalado.
			\end{itemize}
			
			\item {\bf Consultas a la espera de cambios en los datos} \label{prob:8}
			\begin{itemize}
				\item {\bf Solución}: Existen dos tipos de consultas en {\it Firebase}, las que devuelven un solo resultado ({\ttfamily addListenerForSingleValueEvent()}) y las que devuelven varios resultados ({\ttfamily addChildEventListener()}). Éstas últimas se quedan esperando por si se modifican datos de la base de datos.
				La solución sería remover el {\ttfamily listener} al final de su uso ({\ttfamily ref.removeEventListener(originalListener);}) o tener una sola consulta del segundo tipo por actividad.
			\end{itemize}
		\end{enumerate}