%
% ---------------------------------------------------
%
% Trabajo de Final de Grado:
% Author: Gonzalo Jesús García Martín <dracoyue@gmail.com>
% Capítulo: Objetivos
% Fichero: Cap5_Problemas.tex
%
% ----------------------------------------------------
%

\cleardoublepage
\chapter{Problemas}
\label{chap:troubles}

	\begin{enumerate}
		\item {\bf Android Studio no render target}
			\begin{itemize}
				\item {\em Solución}:
					\begin{enumerate}
						\item Asegurarse de que hay un dispositivo real o virtual seleccionado.
						\item Tener una versión de Android seleccionada.
						\item Versiones necesarias para el desarrollo dela aplicación en Android instaladas.
						\item Crear un nuevo Dispositivo Virtual (AVD).
						% Meter dibujo de AVD
						\item reiniciar Android Studio.
					\end{enumerate}
			\end{itemize}
		%
		\item {\bf Fallo al encontrar com.android.support:appcompat-v7:16.+}
			\begin{itemize}
				\item {\bf Solución}: Actualizar en las dependencias del archivo ``build.gradle'', en el directorio ``app'' que está en la raíz del proyecto, la librería a la última versión disponible de forma manual.
			\end{itemize}
		%
		\item {\bf Fallo al encontrar Java}
			\begin{itemize}
				\item {\bf Solución}: Si java ya está instalado, averiguar el directorio. Una vez echo esto necesitas volver a establecer la variable de ámbito indicando la localización correcta.
				Selecciona Iniciar \textgreater Equipo \textgreater Propiedades \textgreater Configuración Avanzada del Sistema.
				Entonces abre la pestaña Opciones Avanzadas \textgreater Variables de Entorno y añade una nueva variable de sistema llamada JAVA\_HOME que redirija a tu directorio JDK, por ejemplo ``C:{\textbackslash}Program Files{\textbackslash}Java{\textbackslash}jdk1.7.0\_21''. 
			\end{itemize}
		\item {\bf Duplicidad en las dependencias de los paquetes}
			\begin{itemize}
				\item {\bf Solución}: Si se tiene un error de construcción compilando sobre archivos duplicados, se puede excluir esos ficheros añadiendo la directiva {\it packagingOptions} al archivo {\it build.gradle} que está en el directorio {\it app}.
				
				\lstinputlisting[float,language=Java,caption={Solución a la duplicidad en build.gradle},label={code:duplicidadbuild}]{Code/duplicidadbuild.gradle}
			\end{itemize}
			
		\newpage
		\item {\bf Porblemas con import android.support.v13}
			\begin{itemize}
				\item {\bf Solución}: Añadir en las dependencias del archivo ``build.gradle'', en el directorio ``app'' que está en la raíz del proyecto, la orden ``compile 'com.android.support:support-v13:21.+' '' de forma manual.
				
				\noindent
				\lstinputlisting[float,language=Java,caption={Solución al problema supportV13},label={code:supportv13}]{Code/support13build.gradle}
			\end{itemize}
			
		\newpage
		\item {\bf Problemas con Gradle}
			\begin{itemize}
				\item {\bf Solución}:
					\begin{enumerate}
						\item En el archivo ``buil.gradle'' que está en el directorio raíz del proyecto añadir la dependencia de forma manual: {\it classpath 'com.android.tools.build:gradle:1.0.0'}.
						\item En el archivo ``buil.gradle'' que está en el directorio ``app'' en la raíz del proyecto añadir dentro de ``release'' de forma manual:
							minifyEnabled false\\
							proguardFiles getDefaultProguardFile('proguard-android.txt'), 'proguard-rules.txt'
					\end{enumerate}
					\noindent
					\lstinputlisting[float,language=Java,caption={Solución en build.gradle},label={code:buildGradle}]{Code/buildgradle.gradle}
					\lstinputlisting[float,language=Java,caption={Solución en app/build.gradle},label={code:appBuildGradle}]{Code/appbuildgradle.gradle}
			\end{itemize}
		
		\newpage
		\item {\bf SDK no encontrado}
			\begin{itemize}
				\item {\bf Solución}: Si Android Studio no encuentra el SDK y está instalado seleccionar {\it Windows \textgreater Preferencias \textgreater Android \textgreater Localización SDK} y establecer el directorio donde se tiene instalado.
			\end{itemize}
	\end{enumerate}