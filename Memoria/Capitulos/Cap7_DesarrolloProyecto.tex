%
% ---------------------------------------------------
%
% Trabajo de Final de Grado:
% Author: Gonzalo Jesús García Martín <dracoyue@gmail.com>
% Capítulo: Desarrollo del Proyecto
% Fichero: Cap7_DesarrolloProyecto.tex
%
% ----------------------------------------------------
%

\cleardoublepage
\chapter{Desarrollo del Proyecto}
\label{chap:developing}

	Se toma contacto con las herramientas a utilizar. Para afianzar y adquirir conocimientos nuevos se implementan diferentes tutoriales orientados a facilitar la ejecución correcta del proyecto. Se especifican los requisitos de la aplicación y se deciden los servicios en la nube que se van a usar.
	Se usa UML\cite{58:UML:online} para crear el diagrama de actividades y flujo; se buscan aplicaciones similares y se implementa una base de datos en los servicios web elegidos. Se crean las actividades básicas de la aplicación para conseguir registro y autenticación de usuarios, además de decir el estilo que tendrá la aplicación.
	
	\bigskip
	Se implementa la actividad donde se mostrarán los contactos de usuario. Pero hay un problema con las consultas y es que son asíncronas, es decir, el código se sigue ejecutando mientras la consulta no ha terminado.
	Introducimos mas usuarios en nuestra base de datos y le añadimos a la aplicación cierta navegabilidad. También se crea la función que comprueba si el dispositivo tiene internet y la encargada de comprobar si un alumno existe cuando un usuario con el perfil de padre se registra. Se adapta el registro para incluir los diferentes perfiles de los usuarios que van a usar la aplicación.
	
	\bigskip
	Se ponen todos los botones en español, se programa el código que recupera las notificaciones de un usuario y se gestionan los colegios desde la base de datos añadiendo información de éstos. Se comprueban los campos en los que el usuario tiene que introducir datos, además de codificar la diferenciación de las notificaciones según el perfil del usuario remitente y se usa una imagen para mostrar los campos obligatorios.
	
	\bigskip
	Se implementan las actividades de ayuda, la que se usa para programar eventos en el calendario y la que usarán los usuarios con perfil de profesor para enviar una notificación a los integrantes de una misma clase, ya sean usuarios con el perfil de padre o con el de alumno. Se añaden los idiomas de inglés y francés, se resuelven los problemas que se derivan de éstos pues al cambiar la orientación del dispositivo,  se recreaba la actividad con el lenguaje por defecto del dispositivo. Esto se solucionó almacenando el idioma seleccionado y recuperándolo al inicio de cada actividad.
	
	\bigskip
	Se gestionan las clases de cada colegio desde la base de datos añadiendo información sobre éstas y se cambia la recuperación de usuarios desde dicha base. Se recuperaban atendiendo al correo de los usuarios, pero como los usuarios con perfil de alumno registrados por otro que tenga un perfil de padre pueden no tener correo, se usa el DNI. Por último se añade la funcionalidad para contraseñas olvidadas, dar de baja una cuenta y salir de la aplicación.