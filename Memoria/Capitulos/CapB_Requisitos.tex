%
% ---------------------------------------------------
%
% Trabajo de Final de Grado:
% Author: Gonzalo Jesús García Martín <dracoyue@gmail.com>
% Capítulo:  Especificaciones, Funcionalidades
% Fichero: CapB_Requisitos.tex
%
% ----------------------------------------------------
%

\cleardoublepage
\chapter{Requisitos}
\label{chap:requirements}

Con la aplicaci\'on ``College App'' para ``Android'' se crear\'a un sistema de comunicaci\'on escolar, para que los padres/madres/tutores legales, profesores y alumnos del centro puedan estar comunicados de forma continua.

\begin{itemize}
	\item Sistema de Registro: Al usuario se le podr\'a pedir los siguientes datos:
		\begin{itemize}
			\item Nombre y Apellidos.
			\item Rol con el que se va a registrar, ya que en funci\'on de un rol u otro se le pedir\'a distintos datos.
			\item N\'umero de tel\'efono.
			\item D.N.I
			\item Nombre del centro con el que va a hacer el registro.
			\item Selecci\'on de clase o asignatura, seg\'un lo que imparta el profesor.
			\item Nombre y Apellidos del alumno.
			\item Clase en la que est\'a el alumno. %Difrernciar 2 alumnos con = nombre y apellidos en la misma clase.
		\end{itemize}
	\item Sistema de ``Login'': Se le pedir\'a al usuario que ingrese su identificaci\'on, es decir, su D.N.I y contrase\~na.
	\item Sistema de comunicaci\'on entre alumnos, profesores y padres/madres/tutores legales: Tanto la direcci\'on del centro como los profesores y padres/madres/tutores legales se podr\'an comunicar entre s\'i con un servicio tipo ``chat'', para informar de alguna incidencia, mensaje o notificaci\'on. Este servicio enviar\'a los mensajes a trav\'es de internet usando notificaciones tipo ``push'', tambi\'en se les permitir\'a mandar documentos a trav\'es de la aplicaci\'on.
	\item Sistema de citas: Se podr\'a concertar citas con los padres/madres/Tutores Legales, as\'i como \'estos podr\'an solicitar reuniones con los profesores, estas citas se podr\'an a\~nadir al calendario de Google. Para esto ambos usuarios deber\'an aceptar la cita.
	\item Lista de contactos, organizada por grupos desplegables, por ejemplo los profesores tendr\'an la lista organizadas por las clases que imparten en el centro en la cual estar\'an los alumnos a los que les d\'e clase.
\end{itemize}

Esto permitir\'a establecer un sistema de comunicaciones entre los profesores, la direcci\'on del centro y los padres de alumnos.

\section{Roles}

	\begin{itemize}
		\item[Alumnos] Se podr\'a comunicar con los profesores, pudiendo mandar citas, circulares y documentos varios.
		\item[Profesores] Se podr\'a comunicar con los alumnos y padres, pudiendo enviar citas, circulares y documentos varios. Los padres en la lista de contactos estar\'an organizados por clases existentes en el centro. Los alumnos se mostrar\'an organizados por materias que impartan a los alumnos, al igual que los padres.
		\item[Padre/Madre/Tutor Legal] Se podr\'a comunicar con los profesores, pudiendo recibir circulares y enviar citas y documentos varios. Los profesores en la lista de contactos estar\'an organizados por materias que dan a los alumnos con los que \'este est\'e relacionado.
	\end{itemize}