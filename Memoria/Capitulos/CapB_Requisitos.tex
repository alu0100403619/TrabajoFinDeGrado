%
% ---------------------------------------------------
%
% Trabajo de Final de Grado:
% Author: Gonzalo Jesús García Martín <dracoyue@gmail.com>
% Capítulo: Especificaciones, Funcionalidades
% Fichero: CapB_Requisitos.tex
%
% ----------------------------------------------------
%

\cleardoublepage
\chapter{Especificación de Requisitos Cap B}
\label{chap:requirementsB}

Con la aplicación \CollegeApp para \textit{Android} se pretende crear un sistema de comunicación escolar, para que los padres/madres/tutores legales, profesores y alumnos del centro puedan estar comunicados de forma continua. Ya que se ha encontrado una carencia en un sector tan importante como es el de la formación.\\
\\ 
La aplicación hará uso de los servicios en la nube usando los servicios de \href{https://www.firebase.com/}{firebase}\cite{6:firebase:online}.
Firebase es una web que proporciona servicios en la nube de forma fácil y segura, con una integración bastante sencilla con las nuevas tecnologías. Ofrece servicios de recuperación y guardado de datos, registro y acceso de usuarios, reglas de seguridad, simulador y análisis de datos entre otros. Los datos almacenado en este servicio no son datos \href{http://es.wikipedia.org/wiki/SQL}{\textit{SQL}}\cite{8:jquery:online}\cite{9:jquery:online}, si no que son datos \href{http://es.wikipedia.org/wiki/JSON}{\textit{JSON}}\cite{7:json:online}.\\
\\
La aplicación tendrá una pantalla de bienvenida con tres pestañas:
\begin{itemize}
	\item Bienvenido: Aquí se mostrará los botones para acceder a la aplicación y para el registro de usuarios.
	\item Ayuda: En esta pestaña se obtendrá una pequeña ayuda sobre la aplicación.
	\item Sobre el Autor: Se mostrará la información sobre el autor de la aplicación.
\end{itemize}
En la pantalla de registro se le pedirá al usuario una serie de datos a cumplimentar según su rol: 
\begin{itemize}
	\item Alumno:
	\begin{itemize}
		\item Nombre.
		\item Apellidos.
		\item Teléfono.
		\item Centro Escolar.
		\item Grupo y Curso al que pertenece: 1A, 2C, 3B... Solo podrá poner un curso y un grupo.
		\item Correo electrónico.
		\item Contraseña.
	\end{itemize}
	\item Profesor:
	\begin{itemize}
		\item Nombre.
		\item Apellidos.
		\item Teléfono.
		\item Centro Escolar.
		\item Grupo y Curso al que pertenece: 1A, 2C, 3B... Podrá poner varios cursos y grupos separados por comas.
		\item Correo electrónico.
		\item Contraseña.
	\end{itemize}
	\item Padre/Madre/Tutor Legal:
	\begin{itemize}
		\item Nombre.
		\item Apellidos.
		\item Teléfono.
		\item Centro Escolar al que asiste el alumno del que es padre.
		\item Correo electrónico.
		\item Contraseña.
		\item Nombre del alumno del que es padre.
		\item Apellidos del alumno del que es padre.
		\item Correo electrónico del alumno del que es padre.
		\item Centro escolar del alumno del que es padre.
		\item Curso y grupo del alumno del que es padre 1A, 2C, 3B... Solo podrá poner un curso y un grupo.
	\end{itemize}
\end{itemize}
En la de acceso se le pedirá su dirección de correo y la contraseña, el usuario tendrá, además, la posibilidad de solicitar que se le recuerde la contraseña por correo.\\
\\
Una vez se accede a la aplicación a través del e-mail y contraseña, ésta se encarga de clasificar al usuario según el rol con el que se halla registrado, rellenando los datos de forma correcta según el perfil seleccionado. A continuación se describirá cada una de las pestañas en sus respectivos roles:
\\
\begin{itemize}
	\item Alumnos:
	\begin{itemize}
		\item Alumnos: Esta pestaña contendrá los datos de los compañeros de clase del usuario.
		\item Profesores: Ésta se rellenará con los datos de los profesores que le dan clase al alumno.
		\item Notificaciones: En ésta se podrán encontrar los nombres de quienes envíen mensajes al usuario.
	\end{itemize}
	\item Profesores:
	\begin{itemize}
		\item Profesores: Aquí podremos encontrar los compañeros que dan clase a los grupos a los que el usuario da clase.
		\item Padres: Esta pestaña se rellenará con los padres de los alumnos a los que el profesor da clase.
		\item Alumnos: Aquí podremos encontrar a los alumnos a los que el profesor da clase
		\item Notificaciones: En ésta se podrán encontrar los nombres de quienes envíen mensajes al usuario.
	\end{itemize}
	\item Padres:
	\begin{itemize}
		\item Padres: En esta pestañ se encontrarán los padres de los compañeros de clase del hijo/s del usuario.
		\item Profesores: Aquí estarán los profesores que dan clase al hijo/s del usuario.
		\item Notificaciones: En ésta se podrán encontrar los nombres de quienes envíen mensajes al usuario.
	\end{itemize}
\end{itemize}

También tendremos una opción para enviar circulares a otros usuarios de la aplicación pudiendo adjuntar archivos. Los padres podrán concertar citas con los profesores y viceversa.
Al seleccionar cualquiera de los usuarios que aparecen en las listas, se lanzará un ``chat'' para que los usuarios se comuniquen entre ellos. Al enviar un mensaje éste se almacenará en los servicios en la nube, en la tabla destinada a los mensajes. Esta tabla contendrá el mensaje, el destinatario y el remitente. Cuando el dispositivo reciba el mensaje, lo borrará de la tabla para no sobrecargarla.\\
\\
Al hacer una selección larga, se podrán ver los datos del usuario que se selecciona.
\\
Todos los datos estarán almacenados en la nube, en concreto en los servicios de ``firebase'' que es propiedad de google.

\section{Sistemas de la Aplicación}

\begin{itemize}
	\item Sistema de Registro: Al usuario se le podrá pedir los siguientes datos, según el rol con el que se registre:
		\begin{itemize}
			\item Nombre y Apellidos.
			\item Rol con el que se va a registrar, ya que en función de un rol u otro se le pedirá distintos datos.
			\item Número de teléfono.
			\item Nombre del centro con el que va a hacer el registro.
			\item Clase y grupo.
			\item E-mail y contraseña.
			\item Nombre y Apellidos del alumno.
			\item Clase en la que está el alumno.
			\item Centro en el que está el alumno.
			\item Curso en el que está el alumno.
		\end{itemize}
	\item Sistema de ``acceso'': Se le pedirá al usuario que ingrese su identificación, es decir, su e-mail y contraseña.
	\item Sistema de comunicación entre alumnos, profesores y padres/madres/tutores legales: Tanto la dirección del centro como los profesores y padres/madres/tutores legales se podrán comunicar entre sí con un servicio tipo ``chat'', para informar de alguna incidencia, mensaje o notificación. Este servicio enviará los mensajes a través de internet usando los servicios de ``firebase'', también se les permitirá enviar documentos a través de la aplicación.
	\item Sistema de citas: Se podrá concertar citas con los padres/madres/Tutores Legales, así como éstos podrán solicitar reuniones con los profesores, estas citas se podrán añadir al calendario de Google. Para esto ambos usuarios deberán aceptar la cita.
	\item Lista de contactos, organizada por grupos desplegables, por ejemplo los profesores tendrán la lista organizadas por las clases que imparten en el centro en la cual estarán los alumnos a los que les dé clase.
\end{itemize}

Esto permitirá establecer un sistema de comunicaciones entre los profesores, la dirección del centro y los padres de alumnos.

\section{Roles}

	\begin{itemize}
		\item Alumnos: Se podrá comunicar con los profesores, pudiendo enviar citas, circulares y documentos varios.
		\item Profesores: Se podrá comunicar con los alumnos y padres, pudiendo enviar citas, circulares y documentos varios. Los padres en la lista de contactos estarán organizados por clases existentes en el centro. Los alumnos se mostrarán organizados por materias que impartan a los alumnos, al igual que los padres.
		\item Padre/Madre/Tutor Legal: Se podrá comunicar con los profesores, pudiendo recibir circulares y enviar citas y documentos varios. Los profesores en la lista de contactos estarán organizados por materias que dan a los alumnos con los que éste esté relacionado.
	\end{itemize}
