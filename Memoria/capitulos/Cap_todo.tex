%
% ---------------------------------------------------
%
% Proyecto de Final de Carrera:
% Author: José Lucas Grillo Lorenzo <jlucas.gl@gmail.com>
% Capítulo: To Do 
% Fichero: Cap_todo.tex
%
% ----------------------------------------------------
%

% Notas:
% 	Evitar coloquialidad
% 	Evitar cuantificadores gratuitos: un poco, muy caro, ...
% 	UNIFORMIDAD
% 	Completitud
% 	Hemos hecho, se ha desarrollado, 
% 	La siguiente Figura NO: la Figura XXX. El Listado XXX.
% 	Cada vez que hagas una afirmación "rotunda", AVÁLALA

\chapter{To Do} \label{chap:to-do} 

\section{Pendiente}

\begin{itemize}
\item Para rodinia pon las referencias (a Rodinia) que aparecen en los últimos Papers. https://www.cs.virginia.edu/~skadron/wiki/rodinia/index.php/Main\_Page
\item -----------------------------------------------------------------
\item Escribir los agradecimientos
\item Terminar y revisar el Capítulo Resultados computacionales (Capítulo 6)
\item Presentación
\item Enviar memoria
\end{itemize}

\section{Hecho}
\begin{itemize}
\item Definir un primer esqueleto de la memoria
\item Añadido Capítulo \ref{chap:optGPGPU}: Optimizaciones en GPGPU
\item Trabajando con texmaker y corrector ortográfico
\item Traducir y resumir el Capítulo de \yacf{} (Capítulo \ref{chap:yacf})
\item Introducción al Capítulo sobre el modelo poliédrico (Capítulo 2)
\item Revisado y adaptado el Prólogo
\item Primer esqueleto del Capítulo de Resultados.
\item En el cap. de Resultados: Introd. a Rodinia, qué es, referencias, características de los códigos para los que presentas resultados
\item Descripción de la plataforma de desarrollo (Verode)
\item Escribir el Capítulo de Motivaciones (Capítulo \ref{chap:motivation})
\item Escribir Optimizaciones en GPGPU (Capítulo \ref{chap:optGPGPU})
\item Escribir objetivos del \ac{PFC} (Capítulo \ref{chap:objetives})
\item Revisar y mejorar el prólogo
\item -----------------------------------------------------------------
\item También en resultados: Añadir referencia al benchmark del EPCC (URL)
\item Descripción de software (versiones de 3 compiladores)
\item -----------------------------------------------------------------
\item Añadir las referencias al documento
\item Añadir a las motivaciones el ``Trabajo relacionado''
\item Añadir al Prólogo, el esquema de la memoria los Capítulos que contiene
\end{itemize}

\section{Revisión final}
\begin{itemize}
%\item Aspell
\item Comprobar los Listados
\item Comprobar referencias perdidas y XXX 
\end{itemize}
