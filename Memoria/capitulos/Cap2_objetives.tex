%
% ---------------------------------------------------
%
% Proyecto de Final de Carrera:
% Author: José Lucas Grillo Lorenzo <jlucas.gl@gmail.com>
% Capítulo: Objetivos 
% Fichero: CapA_objetives.tex
%
% ----------------------------------------------------
%


\chapter{Objetivos} \label{chap:objetives}  

El presente \ac{PFC} ha tenido objetivos centrados en varios aspectos
de las ciencias de la computación. Destacan los referentes a la producción de software, y 
aquellos relacionados con la ingeniería del software, así como los relativos a la 
investigación en materia de compiladores orientados a la \acf{HPC}. A continuación se
enumeran detalladamente algunos de los objetivos que nos propusimos abordar en 
este \ac{PFC}:

\begin{itemize}
\item Adquirir conocimientos básicos sobre conceptos, modelos, técnicas y 
herramientas asociadas con la \ac{HPC}.
\item Involucrarse en un proyecto real de investigación, \accULL{} que ya se encontraba en 
marcha en el momento de la incorporación al mismo de este autor, siendo capaz de aportar 
al proyecto elementos significativamente importantes para el desarrollo del mismo.
\item Introducción al funcionamiento de arquitecturas hardware no-estandard (GPUs, 
máquinas vectoriales, etc).
\item Introducirse en el área de la Computación Heterogénea utilizando 
aceleradores hardware, sus técnicas y herramientas.
\item Adquirir nuevos conocimientos en el área de tecnología de compiladores, aplicándolos 
a técnicas de compilación \acf{StS}.
\item Conocer con alto grado de detalle e implicación el compilador de OpenACC \accULL{} y
ser capaz de depurar sus fallos con relativa soltura.
\item Ser capaz de compilar, ejecutar y recopilar resultados computacionales utilizando diferentes compiladores de OpenACC sobre diferentes plataformas de desarrollo.
\item Participación en un proceso de \textit{release} de software (empaquetado, testing, 
validación, etc).
\item Puesta en práctica de los conocimientos de Ingeniería del Software.
\item Ser capaz de participar activamente en la preparación de al menos un artículo científico remitido para su publicación en congreso o revista internacional 
\cite{Grillo:2013:PGD}.
% PON en las referencias el último artículo enviado a congreso, que se encuentra en revisión.
\end{itemize}
